\documentclass[a4paper,10pt]{article}
\usepackage[utf8]{inputenc}
\usepackage[T1]{fontenc}
\usepackage{url,french}

\sloppy

\begin{document}

\title{\textbf{Proposition de These Paris-Saclay/LRI}}
\author{Safouan Taha and Burkhart Wolff}
\date{}

\maketitle

\subsection*{Titre~:}
\begin{center}
\large Vérification et Simulation des Comportements pour la Sûreté \\ de Fonctionnement des Systèmes Autonomes Critiques
\end{center}

\subsection*{Sujet~:}
Les systèmes autonomes ont la particularité d'être soumis,
en plus des défaillances physiques classiques,
à des erreurs relevant d'interactions avec l'environnement
ou de traitement de données
(par exemple, l'éblouissement ou une reconnaissance de forme erronée).  

Conséquemment, on s'intéresse à des \emph{modèles de comportement}
des systèmes embarqués dans des environnements hostiles
et difficiles à modéliser
dans lesquels le système est censé assurer des propriétés de sécurité
en présence de données erronées ou imprécises.
La norme ISO SOTIF introduit une classification de scenarios
nommés \emph{known safe}, \emph{known unsafe} \emph{unknown unsafe} et
\emph{unknown safe}. 


L'objectif de cette thèse est de trouver des modèles comportementaux
qui sont suffisamment flexibles et \emph{open world}
pour addresser des scénarios \emph{known unsafe} et \emph{unknown unsafe}. 

Une théorie de reference pour étudier les modèles comportementaux est
\emph{concurrent sequential processes} (CSP) qui a été introduit dans un livre en 1978 
par Tony Hoare \cite{Hoare:1985:CSP:3921}, et qui a évolué de manière substantielle
entretemps \cite{BrookesHR84,brookes-roscoe85,roscoe:csp:1998}.
CSP decrit non-déterminisme, communication and synchronisation avec un ensemble
minimal des opérateurs et un ensemble des règles qui représentent en meme temps 
un cadre d'étude théorique, un language de modélisation de comportement, et un framework
de vérification et simulation via un presentations des processus par des automates. 
Par contre, CSP processes sont décrit dans un modèle \emph{fully abstract} et sont conçu d'être \emph{compositionel}. 
Isabelle/HOL-CSP\cite{HOL-CSP-AFP} est une théorie générale de CSP générale
représentée dans le système de modélisation et preuve Isabelle/HOL\cite{nipkow.ea:isabelle:2002}. 
Isabelle/HOL-CSP permet d'exprimer :
\begin{itemize}
\item
  des non-déterminismes non-bornés
  (ce qui correspond au \emph{known unsafe}) ; 
\item
  des classements des événements \emph{open-world}
  (ce qui correspond au \emph{unknown unsafe}) ;
\item
  des patrons de processus (ce qui correspond au \emph{known unsafe}).
\end{itemize}
dans le cadre d'un système  permettant
modélisation, simulation et preuve.
En particulier, il permet de prouver des propriétés sur des patrons des environnements
avec une grande variabilité, des moyens de prouver des relations entre environnements
via raffinement, et la correction correction des abstractions des processus (construit a partir des 
ontologies de domaine)  pertinent pour des simulations.

%, les problemes de surete de comportement
%peuvent etre exprime par la notion de raffinement de processus.
%-- Raffinement plus ?
%-- Simulation plus ?
%-- connection a des systemes ingenieurs Altarica

%La g\'{e}n\'{e}ration d'invariants est une technique clef pour l'analyse des langages
%imp\'{e}ratifs, que ce soit pour l'analyse statique ou la
%g\'{e}n\'{e}ration de tests \`{a} partir du programme (o\`{u} elle est un pr\'{e}-requis \`{a}
%l'\'{e}limination de chemins infaisables) ou pour de la v\'{e}rification d\'{e}ductive
%classique (pour laquelle m\^{e}me une construction automatis\'{e}e partielle
%d'invariants peut faciliter de fa\c{c}on substantielle la t\^{a}che globale de preuve).
%Dans les langages synchrones comme Lustre ou Scade (utilis\'{e}s \`{a} l'\'{e}chelle
%industrielle dans le ferroviaire et l'avionique), ces
%techniques sont cruciales pour une d\'{e}composition des programmes qui permet
%le passage \`{a} l'\'{e}chelle d'autres techniques d'analyse statique.
%
%L'objectif de cette th\`{e}se est de d\'{e}velopper, pour un mod\`{e}le de langage
%imp\'{e}ratif, typ\'{e} et bas\'{e} sur les monades (semblable \`{a} celui pr\'{e}sent\'{e} dans
%\cite{DBLP:conf/pldi/GreenawayLAK14}), un ensemble de tactiques en
%Isabelle/HOL pour construire des invariants qui serviront d'entr\'{e}e
%\`{a} d'autres techniques d'analyse statique de programmes.
%\`{A} partir d'une combinaison d'ex\'{e}cution symbolique, de recherche (heuristique)
%d'abstraction de pr\'{e}dicats, de technologies de parall\'{e}lisation et
%preuve SMT,
%de nouvelles formes de g\'{e}n\'{e}ration d'invariants seront explor\'{e}es et impl\'{e}ment\'{e}es.
%Dans le cas des langages synchrones, des avanc\'{e}es ont \'{e}t\'{e} obtenues r\'{e}cemment
%gr\^{a}ce \`{a} l'outil SMT Kind2 \cite{kind2, kind2Web} de v\'{e}rification pour Lustre
%et son successeur JKind \cite{jkind, jkindWeb} qui utilise des g\'{e}n\'{e}rateurs
%d'invariants bas\'{e}s sur des heuristiques \cite{inv1, inv2} reli\'{e}es entre eux
%par du code d\'{e}di\'{e}, ainsi que des d\'{e}monstrateurs g\'{e}n\'{e}raux
%(Kind and PDR \cite{eng}). M\^{e}me si le fondement th\'{e}orique de ces outils est
%parfois probl\'{e}matique, ce type de syst\`{e}mes a montr\'{e} l'int\'{e}r\^{e}t pratique de cette
%approche. 
%L'accent est mis en particulier sur les invariants qui \'{e}tablissent la faisabilit\'{e}
%des chemins
%d'ex\'{e}cution dans le code plut\^{o}t qu'une preuve compl\`{e}te de la
%correction partielle du code. Ceci suffit pour de nombreuses
%techniques d'analyse statique de programmes ainsi que pour les m\'{e}thodes
%de g\'{e}n\'{e}ration de test structurel al\'{e}atoire \cite{aissat:hal-01655414,aissat:hal-01632902}.
%

\subsection*{Objectifs~:}
\begin{itemize}
\item Etude des modèles  des patrons ``ouverts''  d'environnement 
         appropriés dans la domaine des Automobiles Autonomes, ensemble avec 
         des système de contrôle des Automobiles Autonomes
\item Construction d'un environnement de simulation et verification a base des
         processus CSP pour la domaine  des Automobiles Autonomes
\item Construction d'un environnement d'abstractions correctes des comportements
         pour la simulation
\item Integration de l'environnement dans le système d'ingenieurie AltaRica ou similaire
\end{itemize}

\subsection*{Plan de Travail~:}
\begin{itemize}
\item Modélisation des patrons ``ouverts''  d'environnement des Automobiles Autonomes 
\item Modélisation des patrons ``ouverts'' des système de contrôle des Automobiles Autonomes 
\item Développement d'un système de conversion des processus CSP en Automates
\item Développement d'un système de simulation des processus CSP et leur
         intégration dans le systeme similaire AltaRica 
\item Développement du support d'abstraction des processus a base des ontologies de domaine.
\end{itemize}

\subsection*{Cadre organisatoire~:}
Encadrement dans le laboratoire de recherche en informatique (LRI) dans l'equipe VALS.
Directeur Prof. B. Wolff (habil), Co-encadrement: Dr. Sajouan Taha.
Financement de Thèse: SystemX suivant la convention. 

\bibliographystyle{unsrt}
\bibliography{biblio}

\end{document}
