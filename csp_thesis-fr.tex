\documentclass[a4paper,10pt]{article}
\usepackage[utf8]{inputenc}
\usepackage[T1]{fontenc}
\usepackage{url,french}

\sloppy

\begin{document}

\title{\textbf{Proposition de These Paris-Saclay/LRI}}
\author{Safouan Taha and Burkhart Wolff}
\date{}

\maketitle

\subsection*{Titre~:}
\begin{center}
\large Vérification et Simulation des Comportements pour la Sûreté \\ de Fonctionnement des Systèmes Autonomes Critiques
\end{center}

\subsection*{Sujet~:}
Les systèmes autonomes ont la particularité d'être soumis,
en plus des défaillances physiques classiques,
à des erreurs relevant d'interactions avec l'environnement
ou du traitement des données
(par exemple, l'éblouissement ou une mauvaise reconnaissance de forme).

Dans de tels environnements hostiles et difficiles à modéliser, on s'intéresse à des \emph{modèles de comportement} dans lesquels le système est censé assurer des propriétés de sécurité
malgré la présence de données erronées ou imprécises.
La norme ISO SOTIF introduit une classification de scénarios
en \emph{known safe}, \emph{known unsafe}, \emph{unknown unsafe} \ldots, selon si le scénario est connu lors de la conception du système ou bien découvert lors de la phase de test et si ce scenario ne déstabilise pas le système ou bien provoque sa défaillance.

L'objectif de cette thèse est de trouver des modèles comportementaux
qui sont suffisamment flexibles et \emph{open world}
pour résister à des scénarios \emph{known unsafe} et \emph{unknown unsafe}.

Une théorie de référence pour étudier les modèles comportementaux est
\emph{Concurrent Sequential Processes} (CSP) qui a été introduite dans un livre en 1978
par Tony Hoare \cite{Hoare:1985:CSP:3921}, et qui a évolué de manière substantielle
entre-temps \cite{BrookesHR84,brookes-roscoe85,roscoe:csp:1998}.
CSP décrit le non-déterminisme, la communication et la synchronisation avec un ensemble
minimal d'opérateurs et un ensemble des règles. CSP offre à la fois
un cadre d'étude théorique, un langage de modélisation de comportement, et un framework
de vérification et simulation via une implémentation des processus par des automates.
Les processus CSP sont ainsi décrit dans un modèle abstrait avec une grande \emph{expressivité} et conçu pour être \emph{compositionnels}.

Isabelle/HOL-CSP\cite{HOL-CSP-AFP} est une théorie générale de CSP
implémentée dans l'environnement de modélisation et de preuve Isabelle/HOL\cite{nipkow.ea:isabelle:2002}.
Isabelle/HOL-CSP permet entre autres d'exprimer :
\begin{itemize}
\item
  des non-déterminismes non-bornés
  (ce qui correspond au \emph{known unsafe}) ; 
\item
  une hiérarchisation des événements \emph{open-world}
  (ce qui correspond au \emph{unknown unsafe}) ;
\item
  des patrons de processus (ce qui correspond au \emph{known unsafe}) \ldots
\end{itemize}
dans un cadre permettant modélisation, simulation et preuve.
En particulier, il permet de prouver de manière incrémentale des propriétés sur des d'environnements
avec une grande variabilité et de vérifier la correction des processus par raffinement.

%, les problemes de surete de comportement
%peuvent etre exprime par la notion de raffinement de processus.
%-- Raffinement plus ?
%-- Simulation plus ?
%-- connection a des systemes ingenieurs Altarica

%La g\'{e}n\'{e}ration d'invariants est une technique clef pour l'analyse des langages
%imp\'{e}ratifs, que ce soit pour l'analyse statique ou la
%g\'{e}n\'{e}ration de tests \`{a} partir du programme (o\`{u} elle est un pr\'{e}-requis \`{a}
%l'\'{e}limination de chemins infaisables) ou pour de la v\'{e}rification d\'{e}ductive
%classique (pour laquelle m\^{e}me une construction automatis\'{e}e partielle
%d'invariants peut faciliter de fa\c{c}on substantielle la t\^{a}che globale de preuve).
%Dans les langages synchrones comme Lustre ou Scade (utilis\'{e}s \`{a} l'\'{e}chelle
%industrielle dans le ferroviaire et l'avionique), ces
%techniques sont cruciales pour une d\'{e}composition des programmes qui permet
%le passage \`{a} l'\'{e}chelle d'autres techniques d'analyse statique.
%
%L'objectif de cette th\`{e}se est de d\'{e}velopper, pour un mod\`{e}le de langage
%imp\'{e}ratif, typ\'{e} et bas\'{e} sur les monades (semblable \`{a} celui pr\'{e}sent\'{e} dans
%\cite{DBLP:conf/pldi/GreenawayLAK14}), un ensemble de tactiques en
%Isabelle/HOL pour construire des invariants qui serviront d'entr\'{e}e
%\`{a} d'autres techniques d'analyse statique de programmes.
%\`{A} partir d'une combinaison d'ex\'{e}cution symbolique, de recherche (heuristique)
%d'abstraction de pr\'{e}dicats, de technologies de parall\'{e}lisation et
%preuve SMT,
%de nouvelles formes de g\'{e}n\'{e}ration d'invariants seront explor\'{e}es et impl\'{e}ment\'{e}es.
%Dans le cas des langages synchrones, des avanc\'{e}es ont \'{e}t\'{e} obtenues r\'{e}cemment
%gr\^{a}ce \`{a} l'outil SMT Kind2 \cite{kind2, kind2Web} de v\'{e}rification pour Lustre
%et son successeur JKind \cite{jkind, jkindWeb} qui utilise des g\'{e}n\'{e}rateurs
%d'invariants bas\'{e}s sur des heuristiques \cite{inv1, inv2} reli\'{e}es entre eux
%par du code d\'{e}di\'{e}, ainsi que des d\'{e}monstrateurs g\'{e}n\'{e}raux
%(Kind and PDR \cite{eng}). M\^{e}me si le fondement th\'{e}orique de ces outils est
%parfois probl\'{e}matique, ce type de syst\`{e}mes a montr\'{e} l'int\'{e}r\^{e}t pratique de cette
%approche. 
%L'accent est mis en particulier sur les invariants qui \'{e}tablissent la faisabilit\'{e}
%des chemins
%d'ex\'{e}cution dans le code plut\^{o}t qu'une preuve compl\`{e}te de la
%correction partielle du code. Ceci suffit pour de nombreuses
%techniques d'analyse statique de programmes ainsi que pour les m\'{e}thodes
%de g\'{e}n\'{e}ration de test structurel al\'{e}atoire \cite{aissat:hal-01655414,aissat:hal-01632902}.
%

\subsection*{Objectifs~:}
\begin{itemize}
\item Étude des modèles ``ouverts''  d'environnements
         liés au domaine de l'Automobile Autonome, ainsi que des patrons
         des système de contrôle de l'Automobile Autonome;
\item Construction d'un environnement de simulation et de vérification à base de
         processus CSP pour le domaine  de l'Automobile Autonome;
\item Construction d'un environnement de simulation à base d'abstractions correctes des comportements;
\item Intégration de l'environnement avec le système d'ingénierie AltaRica ou similaire
\end{itemize}

\subsection*{Plan de Travail~:}
\begin{itemize}
\item Modélisation des patrons ``ouverts''  d'environnement de l'Automobile Autonome;
\item Modélisation des patrons ``ouverts'' des système de contrôle dans l'Automobile Autonome
\item Développement d'un système de transformation des processus CSP en Automates
\item Développement d'un système de simulation des processus CSP et leur intégration dans un framework similaire à AltaRica
\item Développement d'un dispositif d'abstraction des processus à partir des ontologies du domaine.
\end{itemize}

\subsection*{Cadre organisatoire~:}
Encadrement dans le laboratoire de recherche en informatique (LRI) dans l'equipe VALS.
Directeur Prof. B. Wolff (HDR), Co-encadrement: Dr. Safouan Taha.
Financement de Thèse: SystemX suivant la convention. 

\bibliographystyle{unsrt}
\bibliography{biblio}

\end{document}
